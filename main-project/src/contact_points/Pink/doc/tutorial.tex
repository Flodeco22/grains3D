\documentstyle[]{article}

\begin{document}

\title{Pinkdev\\
Kit de d\'eveloppement d'op\'erateurs pour le traitement d'images
}

\section{Introduction}

Cet environnement a pour but de faciliter le d\'eveloppement
de vos premiers op\'erateurs de traitement d'image. Il inclut 
notamment des fonctions d'entr\'ee/sortie permettant de lire
et d'\'ecrire des fichiers images en niveaux de gris au format 
{\bf pgm} (Portable Gray Map, un format standard). Il propose \'egalement
une structure de donn\'ees permettant la manipulation des pixels
de l'image, une fois celle-ci charg\'ee en m\'emoire.

Pour la visualisation des images, n'importe quel outil standard
peut \^etre utilis\'e. Nous recommandons 
particuli\`erement {\bf xv}, qui permet en outre la conversion 
de formats de fichiers.

\section{La structure xvimage}

Une image est vue comme un tableau rectangulaire \`a deux 
dimensions de {\it pixels} ou picture elements, 
l'intensit\'e de chaque pixel (son niveau de gris) est cod\'e
sur un octet (unsigned char - valeur entre 0 et 255).

En m\'emoire centrale, une image est stock\'ee dans une structure
de type {\bf xvimage}:

\begin{verbatim}
struct xvimage {
  char *name;
  uint32_t row_size;                    /* Size of a row (number of columns) */
  uint32_t col_size;                    /* Size of a column (number of rows) */
  uint32_t depth_size;                  /* Number of planes (for 3d images) */
  uint32_t time_size;                   /* Number of (2d or 3d) images */
  uint32_t num_data_bands;	        /* Number of bands per data pixel,
					   or number of bands per image, or
					   dimension of vector data, or
					   number of elements in a vector */
  uint32_t data_storage_type;           /* storage type for disk data */
  double xdim, ydim, zdim;              /* voxel dimensions in real world */
  void * image_data;                    /* pointer on raw data */
};
\end{verbatim}

Les pixels sont rang\'es dans ce tableau dans l'ordre
suivant:

\begin{verbatim}
pixel 0 de la ligne 1
pixel 1 de la ligne 1
...
pixel row_size-1 de la ligne 1
pixel 0 de la ligne 2
...
\end{verbatim}

\section{L'acc\`es \`a un pixel}

Il faut d'abord r\'ecuperer l'adresse du tableau contenant les pixels~:

\begin{verbatim}
  ptrimage = (unsigned char *)(image->image_data);
\end{verbatim}

Puis, pour acc\'eder au $i^{\mbox{\`eme}}$ pixel de la
$j^{\mbox{\`eme}}$ ligne~:

\begin{verbatim}
  ptrimage[j * rs + i]
\end{verbatim}

\section{Un exemple}

Voici pour r\'ecapituler une fonction {\bf laddconst} qui 
r\'ehausse d'une valeur constante (sauf en cas de d\'epassement
de la valeur 255) le niveau de gris de tous les pixels d'une
image (source~: \verb|src/lib/laddconst.c|, prototypes~: 
\verb|include/laddconst.h|):

\begin{verbatim}
/* ajoute une constante a une image  - seuil si depassement */
/* Michel Couprie - janvier 1999 */

#include <stdio.h>
#include <stdlib.h>
#include <laddconst.h>
#include <mcimage.h>

/* ==================================== */
int laddconst(struct xvimage * image, /* entree: l'image a traiter */
                                      /* sortie: l'image modifiee  */
              int constante           /* entree: la valeur a ajouter */
             )
/* ==================================== */
{
  int indexpixel;
  unsigned char *ptrimage;
  unsigned long newval;
  int rs, cs, N;

  rs = image->row_size;
  cs = image->col_size;
  N = rs * cs;
  
  /* ---------------------------------------------------------- */
  /* calcul du resultat */
  /* ---------------------------------------------------------- */
  ptrimage = (unsigned char *)(image->image_data);
  for (indexpixel = 0; indexpixel < N; indexpixel++)
  {
    newval = (int)(ptrimage[indexpixel]) + constante;
    if (newval < NDG_MIN) newval = NDG_MIN;
    if (newval > NDG_MAX) newval = NDG_MAX;
    ptrimage[indexpixel] = (unsigned char)newval;
  }

  return 1; /* tout s'est bien passe */
}
\end{verbatim}

Il faut bien s\^ur un programme principal qui appelle cette fonction.
Le r\^ole de ce programme principal peut se limiter \`a lire
une image dans un fichier, appeler la fonction {\bf laddconst}, et 
stocker le r\'esultat dans un autre fichier.

\section{La lecture de fichiers image}

La lecture d'une image dans un fichier au format pgm se fait 
par un appel \`a la fonction {\bf readimage}~:

\begin{verbatim}
  struct xvimage * image;
  char *filename;
  ...
  image = readimage(filename);  
\end{verbatim}

Cette fonction retourne un pointeur NULL si la lecture ne s'est
pas d\'eroul\'ee normalement. La fonction {\bf readimage} est d\'efinie
dans \verb|src/lib/mcimage.c| (prototype dans \verb|include/mcimage.h|).

\section{L'\'ecriture de fichiers image}

L'\'ecriture d'une image dans un fichier au format {\bf pgm} se fait 
par un appel \`a la fonction {\bf writeimage}~:

\begin{verbatim}
  struct xvimage * image;
  char *filename;
  ...
  writeimage(image, filename);
\end{verbatim}

La fonction {\bf writeimage} est d\'efinie
dans \verb|src/lib/mcimage.c| (prototype dans \verb|include/mcimage.h|).

\section{L'allocation d'une structure {\bf xvimage}}

Lors d'un appel \`a {\bf readimage}, une structure {\bf xvimage} est 
automatiquement allou\'ee. Pour allouer une structure
{\bf xvimage} sans appeler {\bf readimage}, on peut utiliser la fonction
{\bf allocimage}~:

\begin{verbatim}
  struct xvimage * image;
  int rs, cs;
  ...  
  image = allocimage(NULL, rs, cs, 1, VFF_TYP_1_BYTE);
\end{verbatim}

Pour lib\'erer la place allou\'ee, on peut utiliser la fonction
{\bf freeimage}~:

\begin{verbatim}
  freeimage(image);
\end{verbatim}

Les fonctions {\bf allocimage} et {\bf freeimage} sont d\'efinies
dans \verb|src/lib/mcimage.c| (prototypes dans \verb|include/mcimage.h|).

\section{Un exemple}

Voici pour terminer un programme principal qui permet 
d'appeler la fonction {\bf laddconst} (source~: \verb|src/com/addconst.c|)~:

\begin{verbatim}
/* appel de laddconst */

#include <stdio.h>
#include <mcimage.h>
#include <laddconst.h>

/* =============================================================== */
int main(int argc, char **argv) 
/* =============================================================== */
{
  struct xvimage * image1;
  int constante;

  if (argc != 4)
  {
    fprintf(stderr, "usage: %s in1.pgm constante out.pgm \n", argv[0]);
    exit(0);
  }

  image1 = readimage(argv[1]);  
  if (image1 == NULL)
  {
    fprintf(stderr, "addconst: readimage failed\n");
    exit(0);
  }
  constante = atoi(argv[2]);

  if (! laddconst(image1, constante))
  {
    fprintf(stderr, "addconst: function laddconst failed\n");
    exit(0);
  }

  writeimage(image1, argv[3]);
  freeimage(image1);

} /* main */
\end{verbatim}

\section{Les repertoires}

doc      : documentation\\
include  : les fichiers entete (.h)\\
obj      : les fichiers objet (.o)\\
bin      : les executables\\
src/com  : les sources des programmes destines a etre lances du shell\\
src/lib  : les sources des fonctions de base (mcimage) et de traitement

\end{document}